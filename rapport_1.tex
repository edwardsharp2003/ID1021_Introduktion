\documentclass[a4paper,11pt]{article}

\usepackage[utf8]{inputenc}

\usepackage{graphicx}
\usepackage{caption}
\usepackage{subcaption}

\usepackage{pgfplots}
\pgfplotsset{compat=1.18} 

\usepackage{minted}

\begin{document}

\title{
    \textbf{Introduction assignment}
}
\author{Edward Sharp}
\date{16-01-25}

\maketitle

\section{the clock test}
    In the first bit of code that is run and presented in the task we are running a script
    which is timing how long it takes to complete two {\tt clock\_gettime()}-calls,
    the call to start and the call to stop taking time. This process is looped 10 times which
    gives us 10 time readings.
    From running the code a number of times, the resolution of the clock seems to be too low to
    actually measure how long the program takes to run. The results are a mix of 100 or 0 $\mu s$,
    suggesting that the results given are not accurate.


    In the second piece of code tested, the same thing is done but a single array access is performed
    as well in the form of the {\tt sum}-variable being updated by adding the current element of {\tt given[]}
    The results seem to be roughly the same. The results given are still a mix of 100 or 0 $\mu s$.


\end{document}
